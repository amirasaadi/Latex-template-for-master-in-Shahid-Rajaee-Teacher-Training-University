\section{مقدمه}
\label{s:introduction}
امروزه با توسعه فناوری اطلاعات و ابزارهای نوین یادگیری و کاهش سهم آموزش رسمی در میان دانش‌آموزان و دانشجویان اهمیت نقش یادگیرنده و ابزارهای یادگیری مشخص می‌شود.
یکی از ابزار‌های نوین چندرسانه‌ای و فیلم‌های آموزشی است.کاهش فشار ذهنی یادگیرنده جزء اولویت‌های اصلی طراح چندرسانه‌ای آموزشی است.
فشار ذهنی می‌تواند از خود موضوع، رسانه انتقال و یا حاصل پردازش های شناختی یادگیرنده باشد.
\\
سنجش میزان بارشناختی ایجاد شده در یادگیرنده این امکان را می‌دهد تا بتوانیم با چیدمان و طراحی مناسب چندرسانه‌ای، کاهش بارشناختی یادگیرنده را مدیریت کنیم.
با ابزارهای مختلفی میتوان بارشناختی را سنجید که هریک مزایا و معایب خاص خود را دارد. در این پژوهش سعی شده است به طور خاص سنجش بار شناختی به وسیله داده‌هایی مختلفی که از چشم کاربر گرفته می‌شود مورد بررسی و مرور قرار گیرد.
\\
ساختار گزارش پیش‌رو با این شرح است در بخش
\ref{s:multimedia}
به اصول طراحی چندرسانه‌ای پرداخته می‌شود، این اصول با آزمایش های تجربی نشان داده شده‌ است که سبب کاهش میزان بارشناختی می‌شود. در بخش 
\ref{s:CognitiveLoad}
مفهوم بارشناختی و حافظه فعال ارائه می‌شود و سپس روش های متداول اندازه‌گیری بارشناختی معرفی خواهد شد.
در بخش 
\ref{s:eye}
به طور ویژه به سنجش بارشناختی به‌وسیله داده‌های چشمی و مغزی پرداخته می‌شود.
و در بخش 
\ref{s:data}
به نحوه پردازش داده‌ها و همچنین موارد مطالعه هریک پژوهش های مرتبط با داده‌های چشمی و مغزی در سنجش بار شناختی بررسی خواهد شد.
در نهایت در بخش
\ref{s:conclusion}
به نتیجه‌گیری و جمع‌بندی آن‌چه در این پژوهش مورد مطالعه قرار گرفته خواهیم پرداخت.

\section{مروری بر پیشنهاده پایان‌نامه}
\subsection{بیان مسئله}
در گذشته ابزارهای اصلی یادگیری متن‌ها و سخنرانی‌ها بودند. اما امروزه با توسعه فناوری اطلاعات و رواج ابزارهای ارتباطی نوین مانند وب و شبکه‌های اجتماعی، نقش یادگیری از طریق چند رسانه‌ای افزایش یافته است. هدف چندرسانه‌ای آسان نمودن یادگیری است و یکی از ویژگی‌های مهم چندرسانه‌ای سادگی و فشار ذهنی کم در هنگام یادگیری آن است. در این پایان‌نامه به طور ویژه چندرسانه‌ای آموزش زبان بررسی خواهد شد.
\\
مدلی که برای شناخت بهتر ذهن طراحی شده است، سه سطح حافظه شامل حافظه حسی، فعال و بلندمدت را در بر می‌گیرد. بار شناختی باری است که بر روی حافظه فعال و در طول یک فرآیند شناختی توسط مواد آموزشی ایجاد می‌شود
\cite{paas2003cognitive}
. این که افراد چگونه از متن و تصویر یاد می‌گیرند، سوال اصلی نظریه بار شناختی در یادگیری چندرسانه‌ای است. از این رو، طراحی مناسب چند رسانه‌ای به منظور کاهش پردازش فرعی، یکی از مسائل مهم محسوب می‌شود.
\\
از آن جا که رابطه‌ای بین بار شناختی و فعالیت‌های مغزی وجود دارد
\cite{yan2009spontaneous}
و برخی از فعالیت‌های مغزی مرتبط با بار شناختی در ناحیه پیشانی رخ می‌دهند، می‌توان آن‌ها را از فعالیت‌های مغزی استخراج نمود
\cite{salmon1996regional}
. تاکنون، معیار‌های متفاوتی برای سنجش بار شناختی مورد بررسی قرار گرفته‌اند و رابطه‌ی آن‌ها با بار شناختی ارزیابی شده است. یکی از ویژگی‌های مشترک اکثر آن‌ها، شخصی و وابسته به فرد بودن آن‌هاست به طوری که برای هر فرد خروجی آن‌ها متفاوت است. در این پایان‌نامه به دید دیگری به مسئله یادگیری به کمک چندرسانه‌ای پرداخته خواهد شد. قصد داریم تا رابطه میان مشخصه‌های یک چندرسانه‌ای آموزش زبان و فعالیت‌های مغزی انداز‌ه‌گیری شده را تجزیه و تحلیل کنیم. محتوای چندرسانه‌ای شامل ویدئو، صدا و مفاهیم موجود در آن است که هر کدام چندین ویژگی دارند و ترکیب تمام و یا بخشی از آن‌ها نیز می‌تواند حالت‌های مختلفی داشته باشد.
\\
برای تعیین مشخصه‌های ویدئو، می‌توان ویژگی‌هایی نظیر موارد زیر را تحلیل محتوای ویدئو عنوان نمود: تغییرات در ویدئو، شامل شدت تغییرهای بین قاب‌ها (فریم‌ها) در زمان‌های مختلف، ساده و یا پیچیده بودن هر فریم به صورت جداگانه به لحاظ شدت روشنایی، رنگ‌ها و مفهوم آن، عدم و یا حضور صدا/گفتار و نیز سرعت آن و مطابقت آن با فریم‌های ویدئو و در دامنه مفاهیم، تعداد و طول هجا‌ها، کلمه‌ها و جمله‌ها، سهولت نحوی و معنایی کلمه‌ها اشاره کرد. از طرفی می‌توان هم‌زمان رخ‌دادن یا جداگانه پخش شدن این‌ها مورد بررسی قرار گیرند. به عبارت دیگر، به‌دنبال الگو‌ها و ویژگی‌هایی در چند رسانه‌ای هستیم که با فعالیت‌های مغزی در مکانی خاص از مغز، ارتباط مشخصی دارند.
\\
اگر بتوان رابطه‌ای میان محتوای‌ چندرسانه‌ای آموزش زبان و فعالیت‌های مغزی پیدا کرد، آنگاه می‌توان بدون سنجش بار شناختی حاصل از چندرسانه‌ای با معیار‌های فیزیولوژیک و یا خودانگارانه، تنها با ویژگی‌های محتوا در رابطه با میزان فعالیت‌های مغزی و بار شناختی حاصل از آن برای فراگیران زبان انگلیسی به عنوان زبان دوم اظهار نظر کرد. این امر به تجزیه و تحلیل لحظه‌ای بار شناختی و یا پیش‌بینی آن نیز کمک می‌کند. همچنین خُبرگان طراحی چندرسانه‌ای خواهند توانست پیش از انتشار محتوای خود آن را به لحاظ سهولت یادگیری و میزان پردازشِ شناختیِ فرعی، محتوایشان را مورد ارزیابی قرار دهند؛ که این امر سبب افزایش کارایی و اثر بخشی محصول‌هایشان می‌شود.
\subsection{اهداف اجرای پایان‌نامه}
با توجه به مسئله  بیان شده برای این پایان‌نامه این اهداف بیان شده‌اند. تجزیه و تحلیل محتوای یک چندرسانه‌ای به لحاظ ویدئو، صدا و محتوای معنایی و یافتن رابطه آن‌ها با فعالیت‌های مغزی. سنجش میزان سهولت یادگیری یک چندرسانه‌ای آموزشی و در نهایت پیدا کردن الگویی از فعالیت‌های مختلف مغزی در نواحی مختلف‌ آن در طول پخش یک چندرسانه‌ای.
\subsection{زمان‌بندی اجرای پایان‌نامه}
بر اساس مسئله و اهداف بیان‌شده مراحل انجام پروژه پایان‌نامه در قالب یک جدول زمان بندی ارائه شده است. جدول 
\ref{tab:Ganttchart}
برنامه زمان‌بندی را نشان می‌دهد. مراحل انجام پایان‌نامه عبارتند از:
\begin{enumerate}
	\item مطالعه روش‌های تحلیل سیگنال‌های فیزیولوژیکی
	\item مطالعه‌ی روش‌های تحلیل ویدئو، صوت و محتوای متنی
	\item آنالیز و تحلیل چندرسانه‌ای‌های تولید شده در راستای استخراج مشخصه‌ها و ویژگی‌های مبتنی بر تصویر، صوت و محتوا
	\item تحلیل و آنالیز داده‌های فیزیولوژیک (سیگنال‌های مغزی و چشمی)
	\item یافتن نگاشت بین مشخصه های چند رسانه‌ای و فعالیت های مغزی (مکانی و زمانی) متناظر
	\item نگارش مقاله
	\item نگارش پایان‌نامه
\end{enumerate}

\begin{table}[h]
	\centering
	\caption{زمان‌بندی انجام پروژه}
	\label{tab:Ganttchart}
	\begin{tabular}{cccccccccccccc}
		\multicolumn{12}{c}{فازبندی (ماه)} &  &  \\
		۱۲ & ۱۱ & ۱۰ & ۹ & ۸ & ۷ & ۶ & ۵ & ۴ & ۳ & ۲ & ۱ & \multirow{-2}{*}{درصد فعالیت} & \multirow{-2}{*}{عنوان مرحله} \\
		\cellcolor[HTML]{000000} & \cellcolor[HTML]{000000} & \cellcolor[HTML]{000000} & \cellcolor[HTML]{000000} & \cellcolor[HTML]{000000} & \cellcolor[HTML]{000000} & \cellcolor[HTML]{000000} & \cellcolor[HTML]{000000} & \cellcolor[HTML]{000000} & \cellcolor[HTML]{000000} & \cellcolor[HTML]{000000} & \cellcolor[HTML]{000000} & ۱۰ & مرحله ۱ \\
		\cellcolor[HTML]{000000} & \cellcolor[HTML]{000000} & \cellcolor[HTML]{000000} & \cellcolor[HTML]{000000} & \cellcolor[HTML]{000000} & \cellcolor[HTML]{000000} & \cellcolor[HTML]{000000} & \cellcolor[HTML]{000000} & \cellcolor[HTML]{000000} & \cellcolor[HTML]{000000} & \cellcolor[HTML]{000000} & \cellcolor[HTML]{000000} & ۱۵ & مرحله ۲ \\
		&  &  &  &  &  & \cellcolor[HTML]{000000} & \cellcolor[HTML]{000000} &  &  &  &  & ۱۰ & مرحله ۳ \\
		&  &  &  &  & \cellcolor[HTML]{000000} & \cellcolor[HTML]{000000} &  &  &  &  &  & ۱۵ & مرحله ۴ \\
		&  & \cellcolor[HTML]{000000} & \cellcolor[HTML]{000000} & \cellcolor[HTML]{000000} & \cellcolor[HTML]{000000} &  &  &  &  &  &  & ۲۰ & مرحله ۵ \\
		& \cellcolor[HTML]{000000} & \cellcolor[HTML]{000000} & \cellcolor[HTML]{000000} &  &  &  &  &  &  &  &  & ۱۵ & مرحله ۶ \\
		\cellcolor[HTML]{000000} & \cellcolor[HTML]{000000} & \cellcolor[HTML]{000000} &  &  &  &  &  &  &  &  &  & ۱۵ & مرحله ۷
	\end{tabular}
\end{table}